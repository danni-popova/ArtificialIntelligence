\section{Literature Review}
Research has been done into multiple ways that robots can navigate \cite{Bonin-Font2008}. Two major ways in which one can categorise navigation is by map-based and mapless navigation. In mapless navigation no prior knowledge of the environment here, much like the IRIS robot there is no prior knowledge of the environment and the robot is expected to navigate and explore based on reactions to sensory inputs.

Reflex agents have also been used as an approach to robotic navigation \cite{Borenstein1989}. In this particular approach the robots direction of travel is determined by virtual attractive and repulsive forces created by the robot's knowledge of the environment created using it's sensors. This is an attractive approach and can be somewhat applied to this project's task as it can take advantage of the sensors to influence the robot's behaviour. However, using a reflex agent requires the ability to define the problem. This is beneficial because you can define how the robot should react, however the downside is the need for a clear understanding of the problem. Other approaches that involve the robot learning are are exciting as they can learn to deal with situations that may not have been envisaged.

Alternatively, if the environment is known there are techniques such as those used in games that could be applied. A relatively well known method used in computer games is the A* algorithm \cite{Cui2011}. The A* algorithm is a searching algorithm similar to Dijkstra's algorithm. The difference is the algorithm has some reference to where it is ultimately going to guide the search in more complex spaces. While this has the drawback of only necessarily working in environments that are mapped out, this can lead to a more direct route than an algorithm that has to explore first.

Further improvements over A* are to restrict the search space such that it will not explore areas that will not lead to a solution \cite{Bjornsson}. Research has proposed heuristics that can rule out areas that will not lead towards the goal. In the task modelled here that could mean accessing other rooms who do not contain the goal, and do not lead anywhere else. Being able to rule these out in advance would significantly reduce the search space.

Artificial neural networks are an exciting prospect for navigation problems \cite{Deb2011}. Opposed to other methods a neural network does not require previous definition of the problem. Instead, you must define a series of inputs and outputs. Then the algorithm should be trained such that the outputs it gives align with the trainer's desires. In a neural network a neuron will multiply an input value by it's weight to create an output. A series of these neurons are connected together to form a the neural network. This would be a beneficial approach as the system can be improved based on it successfully performing the task, therefore complex behaviour could be modelled without the ability to fully define the problem. However, this has a drawback in that it will take time to train, and it is not necessarily possible to predict how it will react, this could be dangerous if used on a robot around members of the public.

Genetic algorithms are an approach of solving a problem in the form of a fitness function my minimizing it's value. This has been applied to neural networks used for pathfinding in drones as if one can define the characteristics they are looking for in the resulting route in the form of a fitness function then the neural networks weights can be honed using a GA \cite{Gautam2014}. This could prove beneficial, as if a neural network were selected a method of training it's weights would be required. The fitness function could score the robots performance, and the chromosome could be the NNs weights.

Again, further research has been done to look into training a neural network for use guiding a robot \cite{Floreano}. This example looks at how practical examples can be used, not just training in a simulation. This has shown that it is possible to use this approach to train a neural network for robot navigation. It again however highlights the issue with this approach. The issues being that because of the nature of the algorithm it is impossible to predict the exact behaviour of the robot when it is presented with different obstacles.

Research has also looked at how a solely a genetic algorithm can be used to produce a route for the navigation of drones \cite{Zheng2005, Zhang2015}. This approach defines the route as a sequence of points before evaluating how fit they are according to length and avoidance of hostile radar amongst other values. This approach is valid and introduces interesting metrics that could enhance our fitness function. Although this approach claims to be capable of obstacle avoidance, the risk with more complex routes is that this would not be able to find an appropriate new route in time whilst it carries out it's evolutions. The benefit of using a GA to train an NN is that the Neural Network can carry out the decision making in near real time, the GA is only used during the training phase.

This algorithm will take advantage of a GA to train a NN. This method has been selected as it gives the best malleability, as once the objectives are laid out the robot simply has to be run and ranked according to how well it meets those objectives. 