\section{Conclusions}
This paper takes a look at approaches to finding a route through an environment, and effective ways to train a robot to navigate. 

The objective of this paper was to find a way to effectively teach a robot to navigate in a foreign environment. This paper has taken steps in the direction of discussing methods for autonomous navigation and discussed the pros and cons of each. It is safe to say that depending on whether a robot is mapless or not can have an impact on the approach taken. For our example where the robot was only required to navigate Anglsea building, the approach of making a reactionary robot without knowledge of the building was unnecessary. A potentially smarter approach would have been to create a robot who had knowledge of the environment and that would only react when presented with an obstacle it did not know about to path around it.

However, with that said this research did successfully show improvement through a number of generations for a robot autonomously navigating and avoiding obstacles in an environment of which it had no prior knowledge.

Future research should look at ways to further combine these efforts. Potentially creating robots that will be able to navigate maplessly, however will also be able to build a map as they navigate and use this to aid future trips to the same area.

\subsection{Contributions}
Both students contributed equally to this project.